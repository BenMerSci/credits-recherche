\documentclass[12pt]{article}
\usepackage{abstract}
\usepackage[T1]{fontenc}
\usepackage[utf8]{inputenc}
\usepackage[numbers]{natbib}
\usepackage[french]{babel}
\usepackage{graphicx}
\usepackage{csvsimple}
\usepackage[onehalfspacing]{setspace}
%\setlength{\parskip}{2em}
\usepackage[compact]{titlesec}
\titlespacing{\section}{0pt}{*0}{*0}
\titlespacing{\subsection}{0pt}{*0}{*0}


\title{Réseaux écologiques et réseaux sociaux : étude comparée}
\author{Benjamin Mercier\\
        \\
        \\
        \\
        Faculté des Sciences,\\
        Département de Biologie,\\
        Université de Sherbrooke
       }
\date{\today}
\setlength{\parskip}{2em}

\begin{document}

\begin{titlepage}
\maketitle
\end{titlepage}

%Permet de mettre des lignes en haut et en bas de l'abstract
\renewenvironment{abstract}
{\begin{quote}
\noindent \rule{\linewidth}{.5pt}\par{\bfseries \abstractname.}}
{\medskip\noindent \rule{\linewidth}{.5pt}
\end{quote}
}

\begin{center}
    \begin{abstract}
      Les propriétés des réseaux sont de plus en plus utilisées à travers une diversité de disciplines scientifiques pour décrire des systèmes complexes,
      ce qui permet d’établir des parallèles intéressants entre l’organisation de systèmes à première vue complètement indépendants.
      Afin d’investiguer la similitude de réseaux sociaux et écologiques, un réseau de collaboration d’élèves de troisième année de baccalauréat en écologie est
      comparé à des réseaux trophiques. Le réseau total présente des caractéristiques des réseaux small-world, le rapprochant du réseau trophique observé dans
      l’estuaire du Ythan. Le sous-ensemble représentant uniquement les membres de la cohorte se trouve davantage représenté par un petit système trophique
       désertique par sa forte connectance.\\
  \end{abstract}
\end{center}


\section{Introduction}

\\



\section{Méthode}

\\


\section{Résultats}






%Insertion du tableau

%\input{fichiers_nets/table_latex.tex}






\pagebreak 


\section{Discussion}


\subsection*{Connectance des réseaux}


\subsection*{Réseau complet}


\subsection*{Réseau des noeuds majeurs}


\section{Conclusion}

\pagebreak

%Le style unsrtnat est bien adapté à Natbib et met la bibliographie en ordre d'apparition dans le texte et non en ordre alphabétique
\bibliographystyle{unsrtnat}
\bibliography{ref}

\end{document}